\chapter{Úvod}
\label{sec:Introduction}
Kryptoměnové trhy, obchodování a celková budoucnost tohoto fenoménu je zajímavým tématem této doby. Vysoká volatilita, jakožto jedna z hlavních charakteristik tohoto trhu, s sebou přináší
možnost interesantních strategií obchodování. Tato práce je zaměřena právě na téma nalezení vhodných příležitostí obchodování kryptoměnových párů na světových burzách. Hlavním cílem
je vytvořit nástroj pro automatizované provádění technické analýzy a obchodování kryptoměn.

Teoretická část práce, která je zároveň první částí, se zaměřuje na obecný popis kryptoměn, co to je kryptoměna, blockchain a jakými způsoby probíhá ověřování validní transakce. Dále jsou
přiblíženy kryptoměnové burzy, jejich historie a poskytované funkce. S burzami se pojí i obchodní příkazy. Rozdíly mezi jednotlivými obchodními příkazy a způsob, jak jsou vykonávány je
taktéž součástí první části této práce.

Druhá kapitola se zabývá technickou analýzou, jenž slouží jako základ pro automatizované obchodování kryptobota. Je zde uvedena historie technické analýzy a hlavní principy,
na které je postavena. Důležitou součástí této kapitoly jsou grafové vzory a technické indikátory. Pro názornost je součástí každého vzoru i indikátoru grafické znázornění.

Následující, třetí, kapitola analyzuje existující kryptoboty a jejich funkce. Vysvětluje, na jakých hlavních bodech kryptoboti staví respektive jak lze kryptoboty rozdělit do několika
kategorií. Shrnují se zde výhody a nevýhody oproti tradičnímu obchodování.

Předposlední kapitola je věnována legislativě. Popisuje se převážně český legislativní rámec kryptoměn, jejich obchodování a danění. Důležitou částí je však taky popis chystané evropské
regulace MiCA. Je zde uvedeno, jak MiCA kategorizuje kryptoaktiva a jednotlivé pravidla pro vydávání a zacházení s nimi.

Finální kapitola je věnována praktické implementační části. Popsán je hlavní princip kryptobota, využité metody technické analýzy a scénáře obchodování. Čtenáři je blíže přiblíženo jak
kryptobot vybírá vhodné páry k obchodování. Součástí je i popis jednotlivých případů opakování obchodních příkazů a skutečné provedení. V rámci kapitoly je uvedeno, jak probíhá komunikace
s burzou Binance. Poslední částí této kapitoly je shrnutí vykonaných obchodů a výdělečnosti kryptobota. Dále je popsáno, jak lze bota rozšířit, jaké funkce lze přidat aby se stal ještě
robustnějším nástrojem pro ziskové obchodování.
\endinput