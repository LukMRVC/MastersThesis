\chapter{Kryptoměny a obchodování na burzách}
\label{sec:CryptoAndTrading}

Fenomén Bitcoinu, který jako první odstartoval kryptoměnový boom, způsobil várku mnoha nových
kryptoměn, založených buďto na podobných principech a technologiích, nebo s novými inovativními myšlenkami.
Spolu s kryptoměnami přišlo spoustu nového názvosloví, byly nimi inspirovány NFT\footnote{Non-fungible tokens} a vzniklo
nové odvětví digitálních finance, označované jako DeFi (decentralizované finance).
Tato kapitola postupně popíše, co to jsou kryptoměny, s využítím jakých technologiích fungují, jak jsou ověřovány
a obchodovány na burzách.


\section{Coiny, chytré kontrakty a tokeny}
\label{sec:CoinsTokensSmartContracts}
Často se lze setkat s pojmy \enquote{token} nebo \enquote{coin}. Obojí se označuje za kryptoměnu a mnohdy se zaměnují
za jednu a stejnou věc. Jak token, tak i coin žijí na blockchainu.
Koncept blockchainu je vysvětlen v následují sekci. Co je to vlastně token nebo coin? Jak již bylo zmíněno, obojí využívá
blockchain, avšak hlavní rozdíl spočívá v tom, co reprezentují. Coin označuje kryptoměnu používanou k uchovávání
nebo směny hodnoty. Jako příklad možné považovat například známí Bitcoin. Token slouží k digitální reprezentaci
aktiva, které se dá směnit pomocí blockchainu. Token může reprezentovat nejen fyzickou věc, jako například zlato, ale i
duševní vlastnictví. Důvodem, proč se tyto pojmy často zaměnují, je že token může reprezentovat \emph{coin} na jiném blockchainu.

Myšlenka chytrých kontraktů poprvné zazněla už v roce 1997. Chytrý kontrakt měl odstranit potřebu důvěry třetí strany, při
jednání, nebo "sjednání smlouvy", s někým druhým. Tímto typickým problémem je skupinové financování (crowdfunding). Prostřednictvím
nějakého poskytovale si kdokoli může vytvořit svůj projekt a požádat veřejnost o pomoc k dosažení finančního cíle. Jesliže se najde dostatek
jednolivců, ochotných přispět na onen projekt a vysbírá se dostatek finančích prostředků, peníze popotují k zadavateli a ten s touto podporou
svůj projekt uskuteční. V opačném případě, kdy se nepodaří splnit minimální finační cíl, měly by se peníze vrátit zpět všem, kteří přispěli.

Chytré kontrakty na blockchainu jsou vlastně malé programy, napsány ve speciálním programovacím jazyce, plnící přesně tuto funkci prostředníka.
Jakmile je jednou chytrý kontrakt publikován, je neměnný a distribuovaný síti. Skutečnost, že je kontrakt distribuovaný
taktéž znamená, že jeho platnost je ověřována všemi členy na daní síti.
Jelikož se jedná o program, jakmile je např. dosaženo nějakého cíle, okamžitě se vykoná finální akce. Z předchozího příkladu veřejného
financování, by se dal chytrý kontrakt naprogramovat tak, aby kontrakt držel všechny přijaté platby dokud není dosaženo minimální částky.
Tento proces je však naprosto transparentní a automatizovaný.


\section{Ověřování transakcí}
\label{sec:BlockchainSecurity}
Hlavní motivace kryptoměn je možnost platit pomocí Internetu bez toho, aniž by platba byla závislá na centrální autoritě, která by měla
být ověřený a důvěryhodný prostředník. Může se však stát, že i tato centrální autorita nesplní své závazky vůči oběma stranám.

V roce 2008 zveřejnil neznámý autor, či skupina autorů, pod názvem Satoshi Nakamoto kryptoměnu Bitcoin a s ní i systém bezpečného
ověřování plateb bez nutnosti centrální autority. Dva nejdůležitější faktory tohoto zabezpečení spočívají v počítáčové kryptografii
a distribuci dat. Co to teda vlastně je blockchain a jak funguje je vysvětleno na následujícím příkladu.

Nechť existují 4 osoby, Alice, Bob, Ctirad a David. Tyto osoby, aby si nemusely pořád předávat peníze, si vedou jednotnou digitální účetní knihu,
pro kohokoli z této čtveřice dostupnou. Pokud má Alice zaplatit Bobovi 20 Kč, zapíšou si údaj o této transakci do účetní knihy. Vždy na
konci měsíce se všichni 4 sejdou a opravdové peníze mezi sebou vymění. V tento moment se naráží na první závažnou chybu v tomto systému.
Účetní kniha je veřejně dostupná a každý do ní může zapsat jakoukoli transakci. Nic nebrání tomu, aby si David do účetní knihy zapsal, že
mu mají všichni zaplatit 10 Kč a tyto záznamy nejdou ověřit a nedá se jim věřit. Řešení této situace spočívá právě v použití kryptografie,
konkrétně digitálních podpisů. Ke každému záznamu bude muset být přiložen podpis o tom, že osobá předávající peníze tuto transakci předem
viděla a schválila. Digitální podpisy se zakládají na dvojici privátního a veřejného klíče. Je nutné privátní klíč udržet v tajnosti,
pouze pro sebe. Samotný podpis můžeme chápat jako funkci, která na základě vstupní zprávy a privátního klíče, vygeneruje číslo o pevné velikosti,
nejčastěji 256 bitů. Výhodou digitálního podpisu je právě to, že je závislý na vstupu. Pokud se jakkoli změní, je vygenerovaný podpis
naprosto odlišný. Aby šlo ověřit, že podpis je skutečně platně podepsaný osobou vlastnící privátní klíč, existuje druhá funkce, schopná
toto ověření provést. Ověření probíhá na základě původní vstupní zprávy, podpisu a veřejného klíče. Výstupem této funkce je pravdivostní
hodnota říkající, zda-li podpis dané zprávy byl vygenerován za použití přidruženého privátního klíče. Tímto je skoro vyřešen problém ověření
pravdivosti transakce v účetní knize. Aby se předešlo falšování pouhým kopírovaním předešlého podepsaného záznamu, přiřadí se k záznamu
transakce taktéž její číselné pořadí.

Nyní nastává další problém. Co se stane, pokud Ctirad nasbírá na účetní knize obrovské dluhy a na konci měsíce se prostě neukáže a uteče?
Pořád je nutná určitá část důvěry. Řešení je jednoduché. Je potřeba mít na úplném začátku knihy záznamy o tom, že všichni 4 dostanou
určitou částku peněz, kterou nejdříve všichni vloží do nějakého tajného trezoru. Dále už jen stačí jednoduše nedovolit nikomu přidávat
nový záznám o transakci, jestliže si už to nemůže dovolit.

Zbývá poslední překážka a to správa o samotnou digitální účetní knihu. Ta někde musí existovat a musí být poskytována všem. To ale pořád
znamená centrální umístění. Ke zbavení se této obtíže a úplné decentralizace, dostane každý účastník svou kopii účetní knihy. V digitálním světě
to znamená, že
každý účastník bude mít nějaké zařízení, na kterém bude mít kompletní kopii digitální účetní knihy a bude zveřejněna všem ostatním účastníkům,
na jakékoliv síti. Nyní, když všichni mají svou vlastní kopii, musí si umět navzájem vyměnovat informace o proběhlých transakcích a to formou
zpráv posílaných po síti. Aby měli všichni účastníci jistotu, že příjmají stejné zprávy a ve stejném pořadí jako ostatní, vyvstává finální
ověřovací krok. Účetní kniha se rozdělí do jednotlivých bloků, obsahující \emph{N} transakcí. Na závěr každého bloku se přidá speciální
číslo --- \emph{nonce}. Přidání \emph{nonce} se řídí určitým pravidlem, které říká, že prvních \emph{M} čísel hashe bloku budou samé nuly.
Hash bloku je vypočten kryptografickou hashovací funkcí ze všech záznamu na zapsaných na bloku a přidaného nonce. Kvůli vlastnostem hashovacích
funkcí, nelze nonce nějak jednoduše vypočíst, ale je nutné jej uhádnout brutální výpočetní silou. Jakmile někdo z účastníků na této síti
transakcí přijde na nonce nějakého bloku, rozešle tento blok s transakcemi a přidaným nonce. Ostatní účastnici ověří platnost bloku
a uloží si ho. Navíc, aby nešlo pořadí bloků zaměňovat, každý nový blok musí v pomyslné hlavičce obsahovat hash předchozího bloku.
Tímto dochází ke zřetězení bloků (odtud název blockchain).

Takto funguje princip ověřování transakcí na základě tzv. proof-of-work. Věří se vždy tomu blockchainu, do kterého bylo dáno nejvíce
výpočtení síly, tzn. tomu nejdelšímu řetězu bloků. V reálném světě účastníkem v nějaké kryptoměnové síti jsou počítače, zvané nody.
Na nodech se ukládá blockchain a jak již avizováno, věří se vždy tomu nejdelšímu řetezu bloků. Svůj vlastní node si může u sebe doma spustit
téměr kdokoli. Pravidla pro přidávání nonce se mohou lišit v závislosti na kryptoměně.

Zde je také vhodné zodpovědět otázku, jak vlastně kryptoměna vzniká. V uvedené analogii si 4 osoby vložili peníze do společného banku.
Ve světě blockchainu je to takzvaným Genesis blokem (někdy také nazýván \enquote{Block 0}). Jako Genesis blok se označuje úplně první
vytěžený, na který všechny ostatní bloky v blockchainu navazují. Těžbou bloků se zde myslí právě nalezení nonce, který se přidává do patičky
bloku. Těžař jako odměnu za vynaložené usílí a výpočetní výkonu dostává kryptoměnu ve formě speciální transakce přidané na konci vytěženého bloku.
Maximální velikost odměny je specifikována protokolem, kterým těžba probíhá a těžaři respektují.

\subsection{Alternativní ověřování --- proof-of-stake}
Ověřování na základě proof-of-work má 2 zásadní nevýhody. První z nich je ta, že odměny na základě těžby bloků nepřímo podněcují centralizaci.
Jelikož vyšší výpočetní výkon znamená vyšší šanci na úspěch při těžbě bloků, stává se, že těžaři se spojují do tzv. \enquote{mining pools}.
V těchto poolech je odměna za vytěžení bloku rozdělena mezi jednotlivé účastníky. Jestliže však mining pool naroste do rozměru, kdy by tvořil
alespoň 51 \% výpočetního výkonu kryptoměnové sítě, teoreticky bude tento pool schopen tvořit nové bloky s falešnými transakcemi. Tato situace
bývá označována jako \enquote{51\% útok} a při provedení tohoto útoku dochází ke kolapsu kryptoměny. K dosažení tohoto útoku je nicméně
nutno mít nemálo prostředků.

% TODO: Vložit citaci na https://ccaf.io/cbeci/index a https://ourworldindata.org/energy/country/czech-republic#how-is-energy-consumption-changing-from-year-to-year

Druhým problémem proof-of-work je vysoká energetická náročnost. V době psaní této práce se odhaduje roční energetická náročnost těžby pouze Bitcoinu
na 88,5 TWh. Pro srovnání, celá Česká republika za rok 2021 spotřebovala okolo 466 TWh. Tato spotřeba zatěžuje jak rozvodné elektrické sítě tak i těžaře.

Převážně z důvodu velké spotřeby elektřiny v roce 2012 byl představen alternativní přístup k ověřování transakcí, označovaný jako proof-of-stake (dále jen PoS).
PoS upravuje tradiční terminilogii, těžaře nahrazuje \emph{validátory} a namísto \enquote{těžby} bloků se bloky \emph{\enquote{razí}}. PoS je
postaven na konsenzu mezi účastníky v síti. Bloky jsou ověřovány náhodně vybranými validátory, kteří jednotlivé transakce ověří a označí za platné.
Tento validovaný blok je následně přidán do blockchainu. Aby se z účastníka stal validátor, stačí se jednoduše do sítě nabídnout a vložit určitou
\enquote{sázku} (stake). Výše této sázky ovlivňuje pravděpodobnost výběru při selekci validátorů k ověření bloku. Pokud by validátor označil
blok obsahující falešné transakce jako platný, je mu část nebo celá vložená sázka odebrána. Tento postih má být motivací, aby validátoři doopravdy
odváděli svou práci správně. Kriticky důležitý krok při ověřování je výběr validátorů. I u této metody ověřování existuje možnost 51\% útoku,
avšak k dosažení je nutné nabídnout alespoň o něco víc než polovinu tržní kapitalizace kryptoměny, čehož není jednoduché dosáhnout.


\section{Burzy}
\label{sec:Exchanges}

\endinput