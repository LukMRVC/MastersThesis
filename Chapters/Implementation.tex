\chapter{Implementace kryptobota}
\label{chap:impl}

% TODO: base/quote currency (podkladová/kotovaná měna) = dopsat toto rozdělení

Hlavní cíl této diplomové práce spočívá v implementaci kryptobota. Tento kryptobot má vytvořené pravidla vycházející z předem provedené technické analýzy.
Pravidla se následně aplikují na reálný trh a bot provádí obchody. Jeho postup a stav lze uživatelský přivětivě pozorovat z webové stránky, případně jeho akce
zastavit, nebo dokonce kompletně opustit tržní pozici. Implementovaný kryptobot je schopen obchodovat na burze Binance, která je jedna z největších a nejpopulárnějších
světových burz.

\section{Fungování kryptobota}
Pro kompletní fungování je zapotřebí několika kroků. Každý z těchto kroků je vysoce důležitý a výpadek jakékoli části může mít kritické dopady. Proto je každá část systému navržena
a implementována co nejrobustněji a počítá s možností nedostupných služeb, případně databáze. Prvním krokem, který bude vzápětí popsán podrobněji se stahování dat. Navazující
je technická analýza. Z analýzy vycházejí příkazy k obchodování. Ty budou taktéž podrobněji popsány. Poslední krok je následná realizace příkazů, komunikace s burzou
a vyhodnocování výsledků obchodování.

\subsection{Historická data}
Úplným základem bota jsou historická data. Naštěstí burza Binance tato data poskytuje bezplatně a volně ke stažení, prakticky komukoli. Lze si stáhnout svíčky,
jednotlivé obchody nebo agregované obchody. Svíčky jsou poskytovaný v granularitách od 1 vteřiny až po celý den. Z tohoto zdroje Binance zpřístupňuje data s jednodenním
zpožděním. Pro potřebu bota v rámci této práce se stahují minutové svíčky se zmiňovaným jednodenním zpožděním. Důvod proč je toto dostačující je uveden dále v této práci.
Náležitě se musí ošetřit situace, kdy není server Binance dostupný nebo chybí data svicek kryptoměnového páru. Pokud k tomuto dojde, systém si zařadí do fronty svých úloh opětovnou
akci stažení potřebných dat. Jakmile jsou svíčky na lokálním serveru, provádí se importování do databáze. Po úspěšném zapsání do databáze následuje technická analýza.

\subsection{Technická analýza}
Fundamentální krok kryptobota je samotné provedení technické analýzy ve spolupráci s backtestingem. Z tohoto vyplývají nastavení obchodních příkazů se prokazuje výhodné
a má cenu je obchodovat. Řešeni vytvořené v rámci je založené na rozboru kopců a dolin vytvořených zavíracími cenami svíček. Kolísání cen nahoru a dolů by ve spojitém grafu
tvořilo čáru, ve kterých lze identifikovat lokální minima a maxima. Právě tyto lokální extrémy tvoří kopce a doliny. Cílem analýzy je pak odhalit kryptoměnové páry s dostatečně
velkým počtem těchto extrémů včetně jejich rozpětí. Jestliže by rozpětí bylo příliš malé, nemusely by být obchody ziskové. Z důvodu, že se takto snaží odhalit dlouhodobě
ziskové páry nevadí, že se stahují a analyzují data s denní zpožděním. Toto řešení není založené na okamžité reakci, nýbrž na vidině dlouhodobé ziskovosti.

% TODO: Obrázek kopců a doliny s vizualizací nástupní a sestupní hrany
Scénář obchodování pro strategii kopců a dolin je následující: v momentě, kdy se dosáhlo minima, cena se odráží a začíná stoupat. Tím je započata nákupní fáze
se snahou uzavřít obchod s co nejnižší cenou. Další stoupání ceny je příznivou situací, během které se zhodnocuje investice. Dosáhnutím lokálního maxima se trend láme a začíná klesat
-- nastává prodejní fáze opět se snahou prodat, tentokrát za co nejvyšší cenu. Konkrétní realizace je pak popsána v sekci \ref{subsec:exchanges-comm}.


\subsection{Výběr párů a realizace obchodních příkazů}
\label{subsec:trade-orders}
Vytvoření obchodních příkazů je starostí předchozího kroku, zabývající se technickou analýzou. Ovšem počet způsobů, jak tyto příkazy vytvořit není pouze jeden a je závislý právě na
nastavení. Pro analýzu kopců a dolin jsou nezbytně důležité 2 parametry: velikost vzestupu a sestupu. Ty zastávají procentuální změnu, o kterou se musí aktuální cena zvednout nebo klesnou,
aby se situace v časovém okně označily jako dolina, respektive kopec. Díky těmto dvěma jednoduchým parametrům lze získat důležité informace o kryptoměnovém páru, konkrétně:
\begin{itemize}
    \item délka kopce,
    \item délka doliny,
    \item ziskovost.
\end{itemize}
Společně z dat získaných ze svíček (počet obchodů, objemy obchodů, \ldots) a statisticky získaných dat (medián a průměr objemu obchodů) se vydolují další rozhodující informace.
Těmi je poměr počtu obchodů, poměr velikost objemu obchodů a počet obchodů vykonaných za den. Během backtestingu se simuluje obchodování o velikosti o mediánu obchodů. Jestliže
algoritmus odhalí například 200 kopců během jednoho dne, znamenalo by to vykonání 200 obchodů. Toto číslo se dá do poměru s celkovým počtem vykonaných obchodů na kryptoměnovém páru.
Obdobně se získá poměr velikosti objemů jednotlivých obchodů. Tyto poměry, společně s celkovým počtem obchodů, jsou důležité z pohledu ovlivňování trhu a zachování likvidity.
Pokud by velikost těchto poměrů byla příliš velká, bylo by obchodování daného kryptoměnového páru riskantní, jelikož každý obchod kryptobota by mohl výrazně ovlivnit situaci na trhu
a způsobovat výkyvy ceny. Zároveň by obchody nemusely být vypořádané, jednoduše protože by se trh stával nelikvidním a rozpětí mezi poptávkou a nabídkou by se rozšířilo.

Informace o délce kopců a dolin slouží jako pomocník pro nastavení maximální doby, po kterou se má bot pokoušet nakoupit. A samozřejmě, ziskovost slouží jako hlavní indikátor
pro výběr páru. Ovšem ziskovost záleží na vybraném stylu obchodování a opakování příkazu, které si zaslouží podrobnější popis.

\subsubsection{Opakování příkazu}
% TODO: Obrázek dělení příkazu na jednotlivé iterace, každé iterace má 2 fáze.
Samotný příkaz se skládá z několika opakování (iterací). Každá iterace má 2 fáze -- nákupní a prodejní. Pro úplnost je ještě nutné definovat, co znamená ukončení jedné iterace.
Ukončení iterace nastává ve 2 případech:
\begin{enumerate}
    \item nepovedla se nákupní fáze (tj. nic se nekoupilo),
    \item ukončila se prodejní fáze.
\end{enumerate}
Prodejní fáze je uzavřena opět ve 2 situacích:
\begin{enumerate}
    \item úspěšný prodej,
    \item překročením maximální doby a prodej příkazem \textsc{market}.
\end{enumerate}

% TODO: Obrázek kaskády vs sekvence
Existují 2 řešené postupy pro opakování příkazů. Prvních z nich je jednoduše v sekvenci, kdy se před začátkem iterace $n_i$ čeká na ukončení iterace $n_{i - 1}$, přičemž $i$ je pořadové
číslo iterace. Výhodou tohoto přístupu je jednoduché sledování aktuálního stavu otevřené pozici na trhu. Avšak nevýhoda je šance na zanedbání dlouhého kopce a nevyužití případné ziskové situace.
Proto byla vymyšlena druhá metoda opakování.

Schopnost iterace v kaskádě s sebou přináší řešení na dlouhé kopce, ve kterých cena roste delší dobu. Kaskáda se oproti sekvenci liší v možnosti spuštění následujícího opakování příkazu
už v momentě, kdy je úspěšně ukončena nákupní fáze. V rámci tohoto může docházet k více otevřeným pozicím na trhu. Nevýhodou je opět eventuální otevření pozice těsně před koncem stoupání
kopce. Jelikož se následně čeká na na procentuální \enquote{propad} ceny dolů, je vysoká pravděpodobnost, že tato pozice bude prodělečná. Další nežádoucí situací je příliš rychlé otevírání
nových pozic a musí se limit buďto časově nebo maximálním počtem otevřených pozic v kaskádě. Bot v tomto případě podporuje obě možnosti.

\subsubsection{Způsoby obchodování}
Implementovaný kryptobot rozlišuje 2 způsoby obchodování, fixní a složené úročení. Fixní způsob otevírá pozice na vždy stejnou obchodovanou částku. Ve své podstatě se jedná o jakýsi DCA. Pokud se
povede utržit zisk, je již \enquote{odložen mimo}. V případě ztráty na obchodě je nutné ji kompenzovat, ať už ze ziskanych zisků nebo ze zásob.
Složené úročení, jak již název napovídá, reinvestuje vše co bylo získáno nebo ztraceno. Účelem je takto maximalizovat zisk. Ovšem tento styl obchodování s sebou přináší riziko ovlivňování trhu. Jestliže
by investovaná částka narostla do velkých rozměrů může obchod způsobit ovlivnění tržní ceny a snížit likviditu trhu. Proto je nutné tento složené úročení limitovat maximální investovanou částkou.


\subsubsection{Skutečné obchodní příkazy}
Obchodování na základě kopců a dolin je pro burzy naprosto irelevantní. Burzy přijímají pouze příkazy popsané v části \ref{subsec:market-trade-orders}. Nezbytností je tedy napasovat scénáře
obchodování do kombinací s těmito příkazy.

Zachycení začátku vzestupného trendu a tedy i začínajícího kopce se dá realizovat jednoduše s použitím příkazu \textsc{trailing stop-limit}. Důležitý
je právě onen \emph{trailing}, který na počítá procentuální změnu tržní ceny. Problémové může být krátkodobé zachvění ceny, které by způsobilo zařazení příkazu do orderbooku, ale ve skutečnosti
by trh pokračoval v klesajícím trendu. Možnost, jak se tomuto vyvarovat nabízí nepovinná \emph{stop} neboli \enquote{aktivační} cena. Teprve v momentě, kdy je proražena stop cena se začíná počítat trailing
delta. Avšak v této situaci nastává komplikace s reálnou burzou: stop cena a limitní cena nelze zadat jako procento aktuálního kurzu, ale musí to být konkrétní hodnoty. Pokud by v nákupní fázi
byla použita stop cena, může dojít k okolnostem, během kterých tržní cena výrazně klesne. Jakmile se odrazí a tržní cena opět stoupne, nedojde k nákupu, neboť se čeká, dokud se neprotne hranice
aktivační ceny. Nejenže tedy by nákup trval dlouhou dobu, ale výrazná část kopce by se \enquote{zaspala} a to rozhodně není žádoucí. Přestože bot poskytuje možnost nakoupit příkazy \textsc{market,
    limit, trailing stop-limit}, skutečně se používá pouze \textsc{trailing stop-limit} bez použití aktivační ceny.

Realizace prodejní fáze se potýká s podobnými problémy. Při použití příkazu \textsc{trailing stop-limit} bez aktivační ceny může dojít k drobnému záchvěvu ceny a předčasným prodejem, což může
vést ke ztrátovému obchodu. Naopak pokud se použije aktivační cena, může dojít k rychlému skoku do klesajícího trendu, aktivační cena nebude nikdy proražena a prodejní fáze bude ukončena na základě
časového limitu s obrovskou ztrátou. Existuje ještě jedna střední cesta a tím je příkaz \textsc{oco} (\enquote{One Cancels the Other}). Jedná se vlastně o kombinaci dvou příkazů, \textsc{limit} a
\textsc{stop limit}. \textsc{oco} příkaz je speciální v tom, že pokud je 1 příkaz jakkoli ukončen, automaticky je zrušen i ten druhý. V tomto případě \textsc{limit} zastřešuje horní hranici, nad
kterou dojde k prodeji, a \textsc{stop limit} spodní hranici v případě pádu tržní ceny. Vytvořený bot umí využít těchto \textsc{oco} příkazů v maximální prospěch periodickým kontrolováním tržní ceny
a vlastním přepočtem procentuální změny. Pokud se cena dostatečně zvýší, zruší momentálně nastavený \textsc{oco} příkaz a zadá nový s cenami posunutými nahoru. K prodeji pak dojde pouze v případě poklesu tržní
ceny na hranici příkazu \textsc{stop limit} nebo pokud by kryptobot zasáhl výpadek. Teto metodě využítí \textsc{oco} příkazů přezdívá autor \emph{peaking}.

\section{Komunikace s burzou}
\label{subsec:exchanges-comm}
Přestože většina logiky kryptobota je realizovaná v databázi pomocí uložených procedur a funkcí, nelze všechno takto provést. Obchodní příkazy kryptobota obsahují drobnou abstrakci, kterou je
nutné převést na reálné hodnoty, především limitní a aktivační ceny. Komunikace s burzou probíhá s pomocí 2 samostatných komponent. Obě mají trochu jiný účel a zaslouží si vlastní krátkou sekci.

\subsection{Executioner}
Komponenta executioner má na starosti obchodní příkazy. Konkrétně jejich zaslání na burzu a jejich zrušení a dohled na vlastní přepočet při použití výše popsané metody \emph{peaking}. Komponenta
využívá asynchronní úlohy, mezi kterými přepíná.

\subsubsection{Zaslání příkazu na burzu}
% TODO: Obrázek cyklu/diagram
Kontrola zaslání příkazu se provádí periodicky, každých pár vteřin. Z databáze (tabulka \emph{trade\_order\_progress}) se načtou potřebné parametry příkazů. Následuje kontrola
dostatku volných prostředků (podkladová měna kryptoměnového páru). Jestliže není dostatek podkladové měny, konkrétní příkaz se neprovede a do databáze se zapíše chybný stav. V případě
úspěchu se pokračuje převedením procentuálních hodnot na reálné částky. Z burzy se skrze API stáhne aktuální kurz kryptoměnového páru, na jehož základě se dopočtou ony reálné částky, které
se chtějí obchodovat. Před závěrečným zasláním příkazu na burzu dochází k poslední kontrole a to na pravidla samotného trhu (minimální nominální hodnota, minimální kvantita, \ldots).
Konečně je příkaz odeslán pomocí REST API ke zpracování na Binance. Součástí odpovědi je pak jednoznačný identifikátor daného příkazu na Binance, ten se uloží do databáze. Tím končí jedna
iterace zasílání příkazu.

\subsubsection{Rušení příkazu}
Po zaslání příkazu na Binance je současně uloženo časové razítko, kdy byl příkaz odeslán. Opět, periodicky, se kontroluje překročení maximální doba trvání příkazu. Pokud se tato doba přesáhne,
je na Binance, přes API, zaslána žádost o zrušení toho příkazu. Pokud se jedná o prodejní příkaz, bot navíc reaguje vznikem nového příkazu pro okamžitý prodej přes \textsc{market}.

\subsubsection{Peaking}
Zvládnutí peakingu s sebou nese starost a vlastní přepočet změny kurzu. Tato změna se počítá od kurzu v momentě zasílání posledního příkazu. Peaking běží v rámci komponenty jako samostatná úloha.
Cyklicky co chvíli si stahuje z Binance momentální kurz kryptoměnového páru a spočítá procentuální přírustek kurzu. Pouze v případě, že se jedná o pozitivní změnu vyšší než zadaná hranice, je
aktuální \textsc{oco} příkaz zrušen a vytvořen nový s posunutými cenami nahoru. I v rámci této rychlé vyměny příkazů se provádí stejné kontroly jako při normálním zasílání přikazu.


\subsection{Stream listener}


\endinput