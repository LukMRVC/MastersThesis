\chapter{Legislativa}
\label{sec:Legislation}
Obchodování na kryptoměnových burzách s sebou nese i tradiční povinnosti jako u jakéhokoliv dalšího zisku. Oproti tradičním akciovým a dluhopisovým trzích nespadají kryptoměny pod žádnou
regulaci nebo centrální autoritu. Tato skutečnost je jak podstatnou kryptoměnou, ale pro denní obchodníky, kteří chtějí udržovat dobré vztahy se svým státem představuje určitou překážku.
Jelikož neexistuje regulace, tak neexistují ani zákony, které dávají jasný směr jak se ziskem utrženým na kryptoměnách zacházet. Následující kapitola se zaměřuje na legislativní problematiku
kryptoměn, chystané zákony a regulace (zejména ČR a EU), případně jejich dopady.


\section*{Česká legislativa}
V čes
Česká národní banka v souvislosti ke kryptoměnám vydala v roce 2018 stanovisko v následujícím znění: \quote{Převodní tokeny nejsou penězi v ekonomickém ani právním smyslu.} %TODO: Citace na https://www.zakonyprolidi.cz/cs/1992-586#f1458009
Dále se v tom vyjádření objevuje, že tokeny nevykazují znaky investičních nástrojů. Tímto se ČNB od kryptoměnového světa dostatečně distancovala. Její povolení k určité činnosti
je vyžadováno pouze ve 3 případech:
\begin{itemize}
    \item obchodování s deriváty na určitý převodní token,
    \item správa majetku investorů, který je investován do převodních tokenů,
    \item provádění převodů pěněžních prostředků v souvislosti s organizaci obchodů s převodními tokeny.
\end{itemize}
Žádné ostatní činnosti, jako například obchodování, směna, i výměna kryptoměny za zboží nepodléha regulaci ČNB.

Ministerstvo financí ČR potvrdilo, že neexistuje žádná legislativa upravující způsob vykazování a účtování digitálních měn. % TODO: citace https://www.financnisprava.cz/assets/cs/prilohy/d-seznam-dani/Info-kryptomeny_priloha1-Sdeleni-MF-k-uctovani-a-vykazovani-.pdf
Tudíž, z právního hlediska, je kryptoměna \emph{nehmotná, zastupitelná, movitá věc}. Tato definice je postačují na daň z příjmu, která je stanovena již na úrovni Evropské unie.
Zde se situace trochu komplikuje. Generální finanční ředitelství již rozlišuje i jakým způsobem byla kryptoměny získána a kdo ji získal, zda právnická či fyzická osoba.
Při vykazování transakcí a nabytí kryptoměny je nutné uvést její získanou hodnotu v Kč, přičemž lze využít přepočtu přes třetí měnu (typicky např. USD, EUR).
Pro účely obchodování je potřeba kryptoměnu nakoupit nebo prodat za fiat měnu. Jelikož kryptoměna není považována za měnu, nejedná se o směnárenskou činnost a tím pádem
se směna daní jako příjem (v případě nákupu jako výdaj) z prodeje nehmotné movité věci.

\section*{Evropská MiCA}


\endinput 