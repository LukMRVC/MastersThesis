\chapter{Legislativa}
\label{sec:Legislation}
Obchodování na kryptoměnových burzách s sebou nese i tradiční povinnosti jako u jakéhokoliv dalšího zisku. Oproti tradičním akciovým a dluhopisovým trzích nespadají kryptoměny pod žádnou
regulaci nebo centrální autoritu. Tato skutečnost je jak podstatnou kryptoměnou, ale pro denní obchodníky, kteří chtějí udržovat dobré vztahy se svým státem představuje určitou překážku.
Jelikož neexistuje regulace, tak neexistují ani zákony, které dávají jasný směr jak se ziskem utrženým na kryptoměnách zacházet. Následující kapitola se zaměřuje na legislativní problematiku
kryptoměn, chystané zákony a regulace (zejména ČR a EU), případně jejich dopady.


\section*{Česká legislativa}
Česká národní banka v souvislosti ke kryptoměnám vydala v roce 2018 stanovisko v následujícím znění: \quote{Převodní tokeny nejsou penězi v ekonomickém ani právním smyslu.} %TODO: Citace na https://www.zakonyprolidi.cz/cs/1992-586#f1458009
Dále se v tom vyjádření objevuje, že tokeny nevykazují znaky investičních nástrojů. Tímto se ČNB od kryptoměnového světa dostatečně distancovala. Její povolení k určité činnosti
je vyžadováno pouze ve 3 případech:
\begin{itemize}
    \item obchodování s deriváty na určitý převodní token,
    \item správa majetku investorů, který je investován do převodních tokenů,
    \item provádění převodů pěněžních prostředků v souvislosti s organizaci obchodů s převodními tokeny.
\end{itemize}
Žádné ostatní činnosti, jako například obchodování, směna, i výměna kryptoměny za zboží nepodléha regulaci ČNB.

Ministerstvo financí ČR potvrdilo, že neexistuje žádná legislativa upravující způsob vykazování a účtování digitálních měn. % TODO: citace https://www.financnisprava.cz/assets/cs/prilohy/d-seznam-dani/Info-kryptomeny_priloha1-Sdeleni-MF-k-uctovani-a-vykazovani-.pdf
Tudíž, z právního hlediska, je kryptoměna \emph{nehmotná, zastupitelná, movitá věc}. Tato definice je postačují na daň z příjmu, která je stanovena již na úrovni Evropské unie.
Zde se situace trochu komplikuje. Generální finanční ředitelství již rozlišuje i jakým způsobem byla kryptoměna získána (těžbou, převodem) a kdo ji získal, zda právnická či fyzická osoba.
Při vykazování transakcí a nabytí kryptoměny je nutné uvést její získanou hodnotu v Kč. Protože pro kryptoměny neexistuje jasný převodní kurz na českou korunu, povoluje se použití přepočtu přes
třetí měnu (nejčastěji např. USD, EUR). V těchto fiat měnách burzy už uvádějí kurz, ke kterým je jednoznačný převod na Kč.   
Pro účely obchodování je potřeba kryptoměnu nakoupit nebo prodat za fiat měnu. Jelikož kryptoměna není považována za měnu, není ani směna kryptoměny za fiat měnu z pohledu
daní z příjmu směnárenskou činností a tím pádem se směna daní jako příjem (v případě nákupu jako výdaj) z prodeje nehmotné movité věci.

% TODO: Pridat citaci https://eur-lex.europa.eu/resource.html?uri=cellar:f69f89bb-fe54-11ea-b44f-01aa75ed71a1.0012.02/DOC_1&format=PDF
\section*{Evropská MiCA}
Chystaná evropská regulace MiCA byla poprvé představena v září roku 2020 Evropskou komisí a měla by vstoupit v planost v roce 2024. Důležitou skutečností je to, že se jedná o regulaci
a měla by tedy předčit zákony členských států EU, zabývající se stejnou problematikou. Dosud si každý členský stát EU mohl vytvořit vlastní zákony a regulace pro obchodování, danění
a práci s kryptoaktivy. Tato volnost přinášela možné mezinárodní podnikatele v branži kryptoaktiv do složitých situací, jelikož pro fungování ve více státech se museli řídit odlišnými
zákony. Hlavní cíle regulace MiCA je posílit regulaci kryptoměnového trhu a poskytnout právní jistotu tohoto trhu v EU. Tohoto chce docílit zvýšení transparentnosti kryptoaktiv a burz,
zavedení větší kontroly na burzami a poskytovatelů digitálních peněženek a v neposlední radě také pojištění investorů. 
MiCA rozděluje kryptoaktiva do 3 kategorií, kterými se zabývá:
\begin{itemize}
    \item \enquote{užitný token},
    \item \enquote{token vázaný na aktiva},
    \item \enquote{elektronický peněžní token (e-token)}.
\end{itemize}
Každé z těchto kategorií se věnují následující podsekce.

\subsection{Užitné tokeny}
Užitné tokeny (angl. \enquote{utility tokens}) jsou definovány jako druh kryptoaktiva, který je dostupný pouze na DLT (nejčastěji tedy blockchain), je přijímán pouze vydavatelem tohoto
tokenu a slouží k poskytování digitálního přístupu ke zboží nebo službě. Jejich hlavní účel tedy není obchodování a placení. Součástí této kategorie jsou tzv. \enquote{governance tokeny},
které umožňují jejich držiteli podílet se na aktivním řízení a hlasování o dalším směru nějakého decentralizovaného/blockchainového projektu (dApps). Hlasování probíhá typicky
použitím chytrých kontraktů popisovaných v předešlé sekci \ref{sec:CoinsTokensSmartContracts}. Hodně těchto tokenů je implementovaných použitím standardu ERC-20 na blockchainu Etherea.
ERC-20 usnadňuje právě tvůrcům tokenů, chytrých kontraktů, peněženek a burzám implementovat nové projekty. Příkladem těchto tokenů je například BAT, využívaný prohlížečem Brave.
BAT je token, který uživatel dostane jako odměnu, když si nechá zobrazit reklamu. Druhým příklad je token MANA (Decentraland) sloužící k nakupování služeb a pozemků ve VR aplikaci
Decentraland.

Podle regulace MiCA mají emitenti těchto tokenů povinnost vydat a zveřejnit \enquote{bílý papír} (whitepaper). Zajímavá je nová povinnost,
kterou MiCA ukládá. Pokud emitent vydá whitepaper, musí autor zahájit svůj projekt do 12 měsíců od jeho zveřejnění. Součástí regulace je spousta dalších povinností, která jsou nad rámec
této práce a proto se autor rozhodl je zde dále nevypisovat. Pro čtenáře lze originální materiál dohledat dle citace.

\subsection{Tokeny vázané na aktiva}
Token vázaný na aktivum kvůli snaze udržet si stabilní hodnotu. Váže se buďto na několik fiat měn, které jsou zákonným platidlem, případně jednu nebo více komodit (plyn, uhlí, ropa, \ldots)
nebo na další kryptoaktiva. Součástí definice je také hodnota kombinace aktiv, tedy vázání hodnoty na určitý fond složený z uvedených možností. Tato kategorie již dostává regulací přísnější
podmínky než předešlá; vydavatel:
\begin{enumerate}
    \item musí mít povolení příslušného orgánu svého domovského členského státu,
    \item musí být právnická osoba usazena v Unii,
    \item vypracuje a zveřejní bílou knihu schválenou příslušným orgánem,
    \item má zodpovědnost za újmu způsobenou uvedením špatných informací.
\end{enumerate}
Jelikož je emitent tohoto tokenu plně zodpovědný za jeho korektní a legální vedení, může držitel těchto kryptoaktiv požadovat odškodné za případnou újmu. Povinností je taktéž
průběžné zveřejňování transparentních informací o množství tokenů v oběhu na webových stránkách a to alespoň jednou za měsíc. Celkově se specifikuje daleko více věcí, jako povinnost
finančních rezerv, řízení rizik a tím zvýšit důvěryhodnost u potenciálních investorů. Bílý papír musí obsahovat varovná sdělení, upozorňující na možná rizika spojená s těmito
tokeny.

\subsection{Elektronické peněžní tokeny}


\endinput