\chapter{Závěr}
Tato práce zpracovala základní informace o kryptoměnách, kryptoměnových burzách a popisuje, jak se dají kryptoměnové
páry obchodovat. Zmiňuje několik typů obchodních příkazů, rozdíly mezi nimi a zjednodušeně vysvětluje fungování obchodních burz. Dále jsou
porovnány 3 strategie obchodování a to scalping, swing trading a buy-and-hold. Následně práce seznamuje čtenáře s technickou analýzou, o kterou
se opírají automatizované obchodní strategie.
Součástí technické analýzy jsou grafové vzory včetně jejich vyobrazení, popisu a vysvětlení významu. Podobně jsou vysvětleny taktéž technické
indikátory s metodami jejich výpočtu.

Další kapitola se zaměřuje na konstrukci kryptobota. Definuje kategorie, do jakých lze kryptoboty řadit podle jejich obchodní strategie. Vysvětluje
základní motivaci a principy, na kterých jsou kryptoboti budováni. V poslední části této kapitoly se nachází srovnání nejpopulárnějších vybraných
kryptobotů.

Předposlední kapitola líčí finanční legislativní rámec z kryptoměn, a to jak z pohledu České republiky, tak Evropské unie. V této části práce je
detailněji rozebrána regulace MiCA, která bude ovlivňovat evropský kryptoměnový trh.

Hlavní kapitolou popisující cíl této práce je realizace a implementace kryptobota. Práce popisuje, jak kryptobot funguje, jaké různé strategie
může využívat a funkce, které kryptobot poskytuje. Mezi tyto patří obchodování v sekvenci nebo kaskádě, respektive investice v podobě fixní nebo složené úročení.
Na základě abstraktně zadaných obchodních příkazů získaných z technické analýzy kryptobot posílá na API reálné burzy požadavky na vytvoření obchodních příkazů.
Je tedy schopný obchodovat na burze, což splňuje hlavní cíl této práce. Navíc je vytvořený bot schopný sledovat aktuální kurzy kryptoměnových párů a dokáže
simulovat tzv. trailingové příkazy nebo zmiňovanou strategie zajištěného obchodování. V závěru práce jsou diskutovány možnosti dalšího rozšíření.


\endinput
