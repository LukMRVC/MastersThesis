\chapter{Automatizované kryptoměnové obchodování}
\label{sec:Cryptobots}
Obchodování na akciových trzích na základě algoritmicky založených botů není žádnou novinkou. Podle zprávy o algoritmickém obchodování na kapitálových trzích v USA z roku 2020 % TODO: Citace https://www.sec.gov/files/Algo_Trading_Report_2020.pdf
se 78 \% obchodů provedlo skrze \enquote{obchodní centra využívají automatizované systémy a algoritmy} (Překlad autora). Není tedy divu, že podobný osud potkal i burzy
kryptoměnové. Ostatně, poskytováním otevřeného přístupu k aplikačním rozhraní se burzy (Binance, Coinbase, \ldots) tomuto zacházení nijak nebrání. Boti se snaží z historických
dat získat cenné informace a s aktuální situací na trhu pracovat ve způsobu, který bude pro jeho uživatele co nejvýnosnější. Oproti člověku, bot nikdy nespí a pracuje
nepřetržitě. Existuje celá řada těchto botů a tato kapitola se zaměřuje právě na boty pracující na kryptoměnových burzách. Nejprve se představí možné výhody,
zanalyzuje jejich způsob fungování, funkce a tvorbu pravidel. V poslední části kapitoly se nachází srovnání několika vybraných kryptobotů.

\section{Kryptoboti}



\endinput