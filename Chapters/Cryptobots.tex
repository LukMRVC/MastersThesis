\chapter{Automatizované kryptoměnové obchodování}
\label{sec:Cryptobots}
Obchodování na akciových trzích na základě algoritmicky založených botů není žádnou novinkou. Podle zprávy o algoritmickém obchodování na kapitálových trzích v USA z roku 2020 % TODO: Citace https://www.sec.gov/files/Algo_Trading_Report_2020.pdf
se 78 \% obchodů provedlo skrze \enquote{obchodní centra využívají automatizované systémy a algoritmy} (Překlad autora). Není tedy divu, že podobný osud potkal i burzy
kryptoměnové. Ostatně, poskytováním otevřeného přístupu k aplikačním rozhraní se burzy (Binance, Coinbase, \ldots) tomuto zacházení nijak nebrání. Boti se snaží z historických
dat získat cenné informace a s aktuální situací na trhu pracovat ve způsobu, který bude pro jeho uživatele co nejvýnosnější. Oproti člověku, bot nikdy nespí a pracuje
nepřetržitě. Existuje celá řada těchto botů a tato kapitola se zaměřuje právě na boty pracující na kryptoměnových burzách. Nejprve se představí možné výhody,
zanalyzuje jejich způsob fungování, funkce a tvorbu pravidel. V poslední části kapitoly se nachází srovnání několika vybraných kryptobotů.

\section{Kryptoboti}
Kryptoměnových botů za odbobí existence tohoto typu trhu již vznikla celá řada. Lze vybírat jak z placených cloudových služeb, tak zdarma dostupných. Dokonce existují i open
source boti, které je možné si zprovoznit na svém vlastním zařízení. Boti z velké části spoléhají na technickou analýzu a technické indikátory popsané v kapitole \ref{chap:TechnicalAnalysis}.
Tyto indikátory vzájemně kombinují, případně si uživatel může sám nakonfigurovat aby bot zaslal na burzu prodejní a nákupní příkazy v momentě, kdy vybrané indikátory
prorazí zadané hranice. Pokročilejší boti umí taky využít sílu strojového učení k predikci tržní ceny. Pro zjišťování optimálních parametrů, případně přeučování modelu
určeného k predikci se využívá \emph{backtesting}. Backtesting spočívá v odhadování úspěšnosti obchodní strategie na základě historických dat. Hlavním cílem celého procesu
je zjistit, jak dobře vybraná strategie funguje v časech, kdy se trh hýbe mezi sestupným a vzestupným trendem, případně když stagnuje.

Automatizované obchodování v sobě skrývá několik výhod. Člověk musí spát. Ovšem Země je velká, když se na jedné části spí, na druhé trh může stále neomezeně obchodovat.
Obchodník se tím nevědomě může ochudit o zisk, nebo dokonce v nejhorším případě celé aktivum se rapidně propadne. Avšak bot nikdy nespí a sleduje trh nepřetržitě. Na změny
dokáže rychle reagovat. V situace kdy vyhodnotí velký propad může se nakoupených aktiv zbavit. Stačí mu pouze předat dostatečnou konfiguraci.

Program se chová systematicky. Má přesně dané pravidla, kterými se řídí. V závislosti na těchto pravidlech se může jednat o velký úspěch nebo naopak ztrátu. Zde lze
najít další výhodu. Každý obchod je reakce na aktuální situaci na trhu a to podle nastavených pravidel. Pravidla se dají zpětně ověřovat a měnit. Tento proces je již
dříve zmiňovaný \enquote{backtesting}. Navíc systematičnost eliminuje lidské emoce. Emoce jsou faktor, který obchodníka význačně ovlivňuje. Technická analýza je pouhá
statistika a na fundament nijak nereaguje. To co člověk může považovat za špatné tržní podmínky může ve skutečnosti znamenat šanci zisku.

Poslední zmíněnou výhodou je obrovská diverzifikace portfolia. Schopnost zvládat obchodovat více aktiv najednou není pro jednotlivce jednoduchý úkol. Zrakový vjem
na to prostě nestačí. Boti mají opět velikou výhodu, jelikož je zajímají jen konkrétní data. A sledovat více různých streamů dat najednou není složité. Jeho uživateli
to pak přináší výhodu ve větší rozmanitosti obchodovaných aktiv, což může v dlouhém měřítku snižovat ztrátu a chránit zisk.

Kryptoboty lze dělit do 3 kategorií:
\begin{enumerate}
    \item trend sledující boti,
    \item DCA (Dollar cost averaging),
    \item skalpovací boti.
\end{enumerate}

\subsection{Trend sledující boti}
První z typu botů je zaměřen charakteristické chování kryptoměn, které se nese ve znamení vysoké volatility. Trh často zažívá velké jak propady, ale i rychlé zvyšování cen.
Právě tohoto střídání trendů se snaží využit trend sledující boti, kteří nakupují v období bull marketu a měnu drží dokud se nezačne trend otáčet. Až v momentě, kdy se objevují
náznaky otočení tržního sentimentu dochází k prodeji a výběru zisku.

\subsection{DCA boti}
Dollar cost averaging je dlouhodobou strategií používající víceméně pouze nákupy. Tyto nákupy se dějí v pravidelných intervalech a s pravidelně investovanou částkou.
Hlavní motivace spočívá v důvěru dlouhodobého výnosu ze získaných kryptoměn, přičemž nákupy se mohou dít v období kdy se trh nachází v cenách jak podprůměrných tak
nadprůměrných. Tím by se měla hodnota zprůměrovat a snížit riziko výrazné finanční ztráty.

\subsection{Skalpovací boti}
Skalpovací boti obchodují v krátkodobých intervalech, jak bylo popsáno v předchozí části \ref{subsubsec:scalping}. Zadávají více příkazů na burzy a výdělek se snaží
získat na mnoha úspěšných obchodech. Provedené obchody nemusí být extra velké ani extrémně výdělečné, postačí i menší zisk. Obchody vznikají především jako důsledek
malých fluktuací tržní ceny. Tyto fluktuace se boti snaží zachytit především použitím technických indikátorů.


\section{Analýza a srovnání existujících kryptobotů}
Za dobu existence funkčních kryptoměnových burz se taktéž hodně rozšířili i kryptoboti. Avšak dost těchto botů je zpoplatněno a nabízí vícero možných plánů, poskytující
pokročilejší funkce. Tato část se více zaměří právě na analýzu a srovnání těchto existujících řešení.

\subsection{Pionex}
Pionex je burza se zabudovanými kryptoboty. Jedním z hlavních lákadel je takzvaný \enquote{grid bot}. Grid bot se dá dobře představit na grafu kryptoměny. Tento bot
vytvoří několik cenových úrovní podle zadaných parametrů nad i pod aktuální tržní cenou. Kdyby se tyto úrovně vykreslily, na grafu by vznikla jakýsi mřížka, odtud název
grid. Jakmile se cena pohybuje pod střední hodnotou, každé proražení nižší cenové úrovně způsobí nákup kryptoměny. Při opačném případě, kdy cena stoupá nad střední hodnotu,
znamená každé proražení cenové úrovně prodej.

Společně s tímto botem obsahuje Pionex dalších 15 ostatních zabudovaných botů. Jejich používání není zpoplatněno, ale Pionex si ukrojí z každého obchodu 0,05 \% jako poplatek.
S touto monetizační politikou může být vhodný pro začátečníky.

\subsection{Cryptohopper}
Tato cloudová služba nabízí napojení na 17 burz, ze kterých si uživatel může vybírat. Obchodní platforma nabízí možnost takzvaného sociálního obchodování. Uživatel si
nemusí vytvářet vlastní signály pro tvorbu pravidel, ale jednoduše po kliknutí \enquote{okopíruje} signál jiného uživatele, který jej dal k dispozici. Bot reaguje na tyto
signály a podle konkrétní konfigurace může kryptoměnu nakoupit nebo prodat. Mimo signály lze také kopírovat kompletní obchodní strategie. K těmto strategiím existuje
obchod nabízející zpoplatněné i neplacené strategie. Strategie je v podstatě kolekce technických indikátorů určující kdy nakoupit a kdy prodat kryptoměnu.

Další možnost automatizace, kterou tato služba nabízí jsou takzvané triggery. Triggery reagují na manuální konfiguraci (kombinace burzy, kryptoměny, indikátoru) a vykonají
přidělené akce. Na výběr z akcí je například jednoduchá notifikace, prodej, nákup nebo opuštění veškerých pozic.

Pro své užvatele nabízí Cryptohopper několik balíků, z čehož odemyká možnosti kryptobota až od placené verze. Disponuje možnosti backtestingu a obchodování \enquote{nanečisto}.
V tomto módu si i nezkušení uživatelé mohou testovat svého bota bez jakéhokoliv ohrožení reálných financí.

\subsection{Coinrule}
Coinrule poskytuje jednoduché a přehledné webové uživatelské rozhraní. Uživatel si propojí účet s nějakou z 11 dostupných burz a začne si tvořit pravidla. Pravidla se tvoří
formou zjednodušeného vizuální programování. Lze přidávat události, podmínky a akce. Události a podmínky reagují na signály vysílané technickými indikátory. Indikátory
si uživatel volí sám, včetně jejích hraničních hodnot. Pokud se zákazník této cloudové služby necítí na tvoření vlastních pravidel, může využít předdefinovaných šablon.

Podobně jako Cryptohopper i Coinrule nabízí několik uživatelských balíků. Je zde však rozdíl v tom, že nabízí neplacený balíček, zpřístupňující sice omezené funkce, ale
všechny nezbytné k automatizaci obchodvání kryptoměn.


\subsection{Shrimpy}
Obchodování na Shrimpy probíhá za pomocí portfolií. Kryptoměny lze do portfolia předat z vlastní kryptoměnové peněženky nebo propojením na nějakou z burz. Shrimpy není navržen jako
skalpovací bot, který reaguje na indikátory. Je zaměřený na dlouhodobý management portfolia, rebalancing\footnote{Rebalancing je jedna z obchodních strategií, ve které
    se aktiva obsažené v portfoliu pravidelně balancují tak, aby hodnotou zabírala stanovou část procenta. Pokud jedno z aktiv nebude zisku, následně se distribuuje mezi ostatní
    aktiva v portofoliu.},
DCA a stop loss. Nabízí taky možnost sociální automatizace ve formě kopírování portfolií.


\endinput