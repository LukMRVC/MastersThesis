\chapter{Automatizované kryptoměnové obchodování}
\label{sec:Cryptobots}
Obchodování na akciových trzích na základě algoritmicky založených botů není žádnou novinkou. Podle zprávy o algoritmickém obchodování na kapitálových trzích v USA z roku 2020 % TODO: Citace https://www.sec.gov/files/Algo_Trading_Report_2020.pdf
se 78 \% obchodů provedlo skrze \enquote{obchodní centra využívají automatizované systémy a algoritmy} (Překlad autora). Není tedy divu, že podobný osud potkal i burzy
kryptoměnové. Ostatně, poskytováním otevřeného přístupu k aplikačním rozhraní se burzy (Binance, Coinbase, \ldots) tomuto zacházení nijak nebrání. Boti se snaží z historických
dat získat cenné informace a s aktuální situací na trhu pracovat ve způsobu, který bude pro jeho uživatele co nejvýnosnější. Oproti člověku, bot nikdy nespí a pracuje
nepřetržitě. Existuje celá řada těchto botů a tato kapitola se zaměřuje právě na boty pracující na kryptoměnových burzách. Nejprve se představí možné výhody,
zanalyzuje jejich způsob fungování, funkce a tvorbu pravidel. V poslední části kapitoly se nachází srovnání několika vybraných kryptobotů.

\section{Kryptoboti}
Kryptoměnových botů za odbobí existence tohoto typu trhu již vznikla celá řada. Lze vybírat jak z placených cloudových služeb, tak zdarma dostupných. Dokonce existují i open
source boti, které je možné si zprovoznit na svém vlastním zařízení. Boti z velké části spoléhají na technickou analýzu a technické indikátory popsané v kapitole \ref{chap:TechnicalAnalysis}.
Tyto indikátory vzájemně kombinují, případně si uživatel může sám nakonfigurovat, aby bot zaslal na burzu prodejní a nákupní příkazy v momentě, kdy vybrané indikátory
prorazí zadané hranice. Pokročilejší boti umí taky využít sílu strojového učení k predikci tržní ceny. Pro zjišťování optimálních parametrů, případně přeučování modelu
určeného k predikci se využívá \emph{backtesting}. Backtesting spočívá v odhadování úspěšnosti obchodní strategie na základě historických dat. Hlavním cílem celého procesu
je zjistit, jak dobře vybraná strategie funguje v časech, kdy se trh hýbe mezi sestupným a vzestupným trendem, případně když stagnuje.
Kryptoboty lze dělit do 3 kategorií:
\begin{enumerate}
    \item trend sledující boti,
    \item DCA (Dollar cost averaging),
    \item skalpovací boti.
\end{enumerate}

\subsection{Trend sledující boti}
První z typu botů je zaměřen charakteristické chování kryptoměn, které se nese ve znamení vysoké volatility. Trh často zažívá velké jak propady, ale i rychlé zvyšování cen.
Právě tohoto střídání trendů se snaží využit trend sledující boti, kteří nakupují v období bull marketu a měnu drží dokud se nezačne trend otáčet. Až v momentě, kdy se objevují
náznaky otočení tržního sentimentu dochází k prodeji a výběru zisku.

\subsection{DCA boti}
Dollar cost averaging je dlouhodobou strategií používající víceméně pouze nákupy. Tyto nákupy se dějí v pravidelných intervalech a s pravidelně investovanou částkou.
Hlavní motivace spočívá v důvěru dlouhodobého výnosu ze získaných kryptoměn, přičemž nákupy se mohou dít v období kdy se trh nachází v cenách jak podprůměrných tak
nadprůměrných. Tím by se měla hodnota zprůměrovat a snížit riziko výrazné finanční ztráty.

\subsection{Skalpovací boti}
Skalpovací boti obchodují v krátkodobých intervalech, jak bylo popsáno v předchozí části \ref{subsubsec:scalping}. Zadávají více příkazů na burzy a výdělek se snaží
získat na mnoha úspěšných obchodech. Provedené obchody nemusí být extra velké ani extrémně výdělečné, postačí i menší zisk. Obchody vznikají především jako důsledek
malých fluktuací tržní ceny. Tyto fluktuace se boti snaží zachytit především použitím technických indikátorů.

\endinput