\chapter{Technická analýza}
\label{chap:TechnicalAnalysis}
Dělat informovaná rozhodnutí je podstata každého obchodníka. V akciových trzích se významně používá fundamentální analýza používaná pro budoucí odhad
chování trendu trhu. Fundament se zabývá vnitřní hodnotou trhu, která je spjatá s obchodními výsledky podniků obchodovaných na burze. Tyto dvě faktory,
tedy dlouhodobé obchodní výsledky podniků často korelují právě s hodnotou akcií. Vnitřní hodnota trhu je pak odhadována za momentální hodnotou jmění
těchto firem. Odhady lze dělat z veřejně dostupných dat, například ze zveřejněných účetních uzávěrek, vyplacených dividend, nové produkty nebo různých ekonomických zpráv.
Oproti tomu, technická analýza se snaží odhadnout budoucí pohyb tržních cen aktiv (forex, akcie, kryptoměny) na základě dostupných historických dat, převážně
z ceny a objemu minulých uzavřených obchodů. Momentální kapitola se podrobněji věnuje technické analýza, její historií, z čeho technická analýza vychází
a zaobírá se několika prostředcích používané při této metodě předpovědi trhu.


\section{Historie}
První náznaky principů technické analýzy se začaly objevovat už v 17. století v práci Josepha de la Vegy o holandských trzích. O století později začíná
rozvoj dalších metody v Asii, která se postupně vyvíjí v použití svíčkové technicky. Dnes se jedná o nejpoužívanější nástroj pro vytváření finančních grafů.
V první půlce 20. století vznikají první knihy zaměřené na technickou analýzu trhu. V té době se jedná převážně o techniky zaměřené na analýzy trendu a
grafové vzory, jelikož výpočetní síla počítačů pro statistickou analýzu nebyla dostupná. Některé z těchto knih, převážně práce Roberta D. Edwardse a Johna
Mageeho \emph{Technical Analysis of Stock Trends}, se považují za klíčové v tomto oboru a zůstávají platné dodnes. V posledních dekádách bylo vytvořeno
mnoho dalších technických nástrojů a teorií, čím dál více opírající se počítačové podpořené výpočetní techniky.


\section{Principy}
Techničtí analytici věří ve 3 hlavní principy. První z nich zní takto: \emph{Ceny se pohybují v trendech}. Tedy, že tržní cena buďto roste, klesá, nebo se pohybuje
pouze do strany přičemž poslední varianta je též označována jako stagnace. Druhý z těchto principů je: \emph{Historie má tendenci se opakovat}. Pokud se historie
opakuje, musí v grafech, které vizualizují vývoj trhu, existovat jisté vzory předpovídající nastávající trend. Posledním principem je tvrzení, že \emph{tržní očekávání
    je reflektováno a započteno na hodnotě aktiva}. Pokud existují novinky, předpovídající, nebo naznačující vzestup například zemědělského trhu díky úrodné sezóně, je toto
očekávání již pozitivně reflektováno na hodnotě akcií zemědělských firem a lze to tím pádem vyčíst i z grafů. Jelikož vzory hrají velkou roli v oboru technické analýzy
je následují sekce zaměřená na popis, identifikaci a význam několika vzorů.


\section{Grafové vzory}
\label{sec:ChartPatterns}
Grafové vzory jsou opakovaně formující se útvary a v technické analýze jsou využívané jako signály buďto přetrvávání nebo obratu momentálního trendu. Vzory se formují při vykreslení
tržních cen na grafu. Typickým a nejčastěji používaným grafem, jak již bylo zmíněno, je graf svíčkový, vyobrazený na obrázku \iffalse TODO: Pridat referenci na obrazek grafu \ref{} \fi.
Vzory lze pak
pozorovat na posloupnosti několika po sobě jdoucích svíček. Důležitým faktorem při rozpoznávání vzorů je zobrazené časové rozpětí, které investor pozoruje. Ne vždy mohou
být vzory dobře rozpoznatelné lidským okem. Užitečnost vzoru časem klesá. Čím později je detekován, tím je méně použitelný.
Existují pokusy použít strojové učení pro detekci těchto vzorů, jako například práce autorů \emph{Marc Velay a Fabrice Daniel}, kterým
se podařilo dosáhnout 97\% recallu, ale model uměl detekovat pouze 1 vzor.
% TODO: Přidat citaci


\subsection{Head and shoulders}
První a zároveň z jeden více známých obratových vzorů je \enquote{Head and shoulders} (Hlava a ramena), viditelný na obrázku \iffalse TODO: Dat obrazek \fi. Jak název napovídá, formace svíček tvoří podobu ramen a hlavy způsobenou
3 kopci a 2 dolinami z čehož je prostřední kopec vyšší než zbývající krajní. V dolinách se cena zastaví na přibližně stejné hodnotě, tvořící tzv. \enquote{neckline} neboli krk.
Během formace levého ramene a hlavy se při stoupající ceně taktéž obchoduje ve vyšších objemech než při klesání. V průběhu tvorby pravého ramene objem obchodů začíná klesat společně s cenou.
Jakmile cena prorazí \emph{neckline} a začne padat pod její úroveň, je vzor jednoznačně dokončen a očekává se pokles ceny. Reálně nemusí formace vypadat naprosto ideálně. Často
může být jedno rameno o něco nižší, případně i širší, než druhé a neckline nemusí být skvěle zarovnaná.


\subsection{Cup and handle}
Formace připomínající šálek s rukojetí (obrázek \iffalse TODO: Ref na obrazek \fi) je symbolem pro rostoucí trend na trhu. Identifikovatelný je tvorbou široké, ale ne příliš hluboké doliny, přičemž na končící ceně této doliny dojde k
dalšímu, již méně razantnímu poklesu v ceně. Tento menší pokles připomíná právě onu rukojeť a musí vždy následovat až po šálku. Toto pořadí je nezaměnitelné. Dolina by měla své dno
zaoblené a připomínat například misku. Šálek ve tvaru \enquote{V} není platným ukazatelem nastávajícího vzoru. Objem obchodů je relativně nizky, ale v období tvorby rukojeti se rapidně zvyšuje.



\subsection{Double top a Double bottom}



\section{Trendové čáry}
\label{sec:TrendingLines}


\section{Indikátory}
\label{sec:Indicators}

\section{Využití}

\section{Výběr krytoměnových párů}
\label{sec:ChoosingCryptopairs}

\endinput